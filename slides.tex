\documentclass[t]{beamer}
% \usetheme{Luebeck}
% \usetheme{Malmoe}
\usetheme{default}

\usepackage{hyperref}
\usepackage[utf8]{inputenc} % this is needed for german umlauts
\usepackage[english]{babel} % this is needed for german umlauts
\usepackage[T1]{fontenc}    % this is needed for correct output 
                            % of umlauts in pdf
\usepackage{xcolor, colortbl}
\usepackage{listings}

\setbeamersize{text margin left=10pt}

\title{Observable}
\subtitle{Geração de observáveis através da metaprogramação}
\subject{C++}
\date{}

\begin{document}

  \lstset{language=C++,
    basicstyle=\small,
    keywordstyle=\color{blue}\ttfamily,
    stringstyle=\color{red}\ttfamily,
    commentstyle=\color{gray}\ttfamily,
    morecomment=[l][\color{magenta}]{\#}
  }

\begin{frame}
  \titlepage
\end{frame}

\begin{frame}[fragile]
  \frametitle{Problema}
  Observar a alteração de estado de objetos C++

  \begin{itemize}
  \item<1->{\textit{Observer pattern (GoF)}}
    \begin{itemize}
    \item<2->{Modo arcaíco: classes base \textit{Observable} e \textit{Observer}}
      \begin{itemize}
      \item<3->{Poliformismo dinâmico} 
      \item<4->{Notificação abrangente (não sei exatamente o que mudou)} 
      \end{itemize}
    \item<5->{Modo moderno: sinais e slots}
      \begin{itemize}
      \item<6->{Poliformismo estático} 
      \item<7->{Notificações específicas (sinais customizados))} 
      \end{itemize}
    \end{itemize}
  \end{itemize}
\end{frame}

\begin{frame}[fragile]
  \frametitle{\textit{Observed}}
  Classe que é observada

  \begin{lstlisting}[escapeinside=`']
    using skills_t = std::map<
      std::size_t, /*skill level*/
      std::string  /*skill name*/
    >;

    struct person_t {
      std::string name;
      std::size_t age;
      skills_t skills;
    };
  \end{lstlisting}
\end{frame}

\begin{frame}[fragile]
  \frametitle{\textit{Boost.Signals2}}
  Uma solução usando \textit{Boost.Signals2}

  \begin{onlyenv}<1>
  \begin{lstlisting}[escapeinside=`']
    struct person_t {
      std::string name() const noexcept
      { return _name; }
      
      void name(std::string val)
      {
        _name = val;
        _name_changed(_name);
      }
    private:    
      std::string _name;
      boost::signals2::signal<void(std::string)>
        _name_changed;
    };
  \end{lstlisting}
  \end{onlyenv}

  \begin{onlyenv}<2->
  \begin{lstlisting}[basicstyle=\tiny]
    struct person_t {
      std::string name() const noexcept
      { return _name; }
      
      void name(std::string val)
      {
        _name = val;
        _name_changed(_name);
      }

      std::size_t age() const noexcept
      { return _age; }
      
      void age(std::size_t val)
      {
        _age = val;
        _age_changed(_age);
      }
    private:    
      std::string _name;
      std::size_t _age;

      boost::signals2::signal<void(std::string)> _name_changed;
      boost::signals2::signal<void(std::size_t)> _age_changed;
    };
  \end{lstlisting}
  \end{onlyenv}
\end{frame}

\begin{frame}[fragile]
  \frametitle{\textit{Boost.Signals2}}
  Notificação abrangente

  \begin{lstlisting}[basicstyle=\tiny,escapeinside=`']
    struct person_t {
      std::string name() const noexcept
      { return _name; }
      
      void name(std::string val)
      {
        _name = val;
        _name_changed(_name);
        `\alert{\_any\_changed();}' 
      }

      std::size_t age() const noexcept
      { return _age; }
      
      void age(std::size_t val)
      {
        _age = val;
        _age_changed(_age);
        `\alert{\_any\_changed();}' 
      }
    private:    
      std::string _name;
      std::size_t _age;

      boost::signals2::signal<void(std::string)> _name_changed;
      boost::signals2::signal<void(std::size_t)> _age_changed;
      `\alert{boost::signals2::signal<void()> \_any\_changed;}'
    };
  \end{lstlisting}
\end{frame}

\begin{frame}[fragile]
  \frametitle{\textit{Boost.Signals2}}
  Containers da STL?

  \begin{lstlisting}[basicstyle=\tiny,escapeinside=`']
    struct person_t {
      //... setters e getters para name e age ...      
      `\alert{skills\_t::iterator insertSkill(std::size\_t level, std::string name)}'
      `\alert{\{}'
      `\alert{  auto it = \_skills.emplace(level, name).first;}'
      `\alert{  \_skill\_inserted(it);}'
      `\alert{  return it;}'
      `\alert{\}}'
    private:    
      std::string _name;
      std::size_t _age;
      `\alert{skills\_t \_skills;}'

      boost::signals2::signal<void(std::string)> _name_changed;
      boost::signals2::signal<void(std::size_t)> _age_changed;
      boost::signals2::signal<void()> _any_changed;
      `\alert{boost::signals2::signal<void(skills\_t::iterator)> 
        \_skill\_inserted;}'
    };
  \end{lstlisting}

  \begin{itemize}
    \item<2->{API proprietária para acessar o container}
    \item<3->{Existem 6 sobrecargas para unordered\_map::insert() em C++11!}
    \item<4->{Alteração no elemento notifica o dono do container? (person\_t)}
    \item<5->{Elementos observáveis?}
  \end{itemize}
\end{frame}

\begin{frame}[fragile]
  \frametitle{\textit{Observable}}
  Biblioteca para construir \textit{Observables} em tempos de compilação usando sinais e slots
  \begin{itemize}
  \item<2->{Elimina código \textit{boilerplate} para a construção do \textit{Observable}}
    \begin{itemize}
    \item<3->{Agilidade na implementação}
    \item<4->{Evita erros de uma solução \textit{handwritten}}
    \end{itemize}
  \item<5->{Não intrusivo. \textit{Observed} está separado do \textit{Observable}}
  \item<6->{Suporte a containers da STL}
  \end{itemize}
\end{frame}

\begin{frame}[fragile]
  \frametitle{\textit{Observable - person\_t}}
  \begin{lstlisting}[basicstyle=\small]
    OBSERVABLE_CLASS_GEN(
      observable_person,
      person_t,
      ((std::string, name))
      ((std::size_t, age))
      ((skills_t, skills))
    )
  \end{lstlisting}

  \begin{onlyenv}<2->
  \begin{lstlisting}[basicstyle=\small]
    inline observable_person 
    observable_factory(person_t& observed) {
      return observable_person(observed, 
                               observed.name, 
                               observed.age,
                               observed.skills);
    }
  \end{lstlisting}
  \end{onlyenv}
\end{frame}

\begin{frame}[fragile]
  \frametitle{\textit{Observable - person\_t}}
  \begin{lstlisting}[basicstyle=\small,escapeinside=`']
    person_t observed{"maria", 26};
  \end{lstlisting}

  \begin{onlyenv}<2->
  \begin{lstlisting}[basicstyle=\small]
    auto operson = observable_factory(observed);
  \end{lstlisting}
  \end{onlyenv}

  \begin{onlyenv}<3->
  \begin{lstlisting}[basicstyle=\small]
    operson.on_change<name>(
        [](std::string name)
        {
          std::cout << "name_has_changed_to_" << name;
        });
  \end{lstlisting}
  \end{onlyenv}

  \begin{onlyenv}<4->
  \begin{lstlisting}[basicstyle=\small]
    operson.assign<name>("MARIA");
  \end{lstlisting}
  \end{onlyenv}
\end{frame}

\begin{frame}[fragile]
  \frametitle{\textit{Observable - person\_t}}

  \begin{onlyenv}<1->
  \begin{lstlisting}[basicstyle=\small]
    auto& oskills = operson.get<skills>();
  \end{lstlisting}
  \end{onlyenv}

  \begin{onlyenv}<2->
  \begin{lstlisting}[basicstyle=\small]
    oskills.on_insert(
        [](const skills_t&, skills_t::const_iterator it)
        {
          std::cout << "new_skill_was_added:_"
                    << it->second << "_with_level_"
                    << it->first;
        });
  \end{lstlisting}
  \end{onlyenv}

  \begin{onlyenv}<3->
  \begin{lstlisting}[basicstyle=\small]
    oskills.emplace(8, "woodworking");
  \end{lstlisting}
  \end{onlyenv}
\end{frame}

\begin{frame}[fragile]
  \frametitle{\textit{Tipos de \textit{Observables}}}
  \begin{itemize}
    \item<1->{Containers da STL}
      \begin{itemize}
      \item<1->{map, unordered\_map, unordered\_set e vector}
      \item<2->{Os elementos são \textit{observables}}
      \item<3->{Não guardam \textit{observables}}
      \end{itemize}
    \item<4->{Classes}
      \begin{itemize}
      \item<4->{Os membros são \textit{observables}}
      \end{itemize}
    \item<5->{Variant}
      \begin{itemize}
      \item<5->{boost::Variant}
      \item<6->{Visitor visita \textit{observable} do elemento}
      \end{itemize}
    \item<7->{Valor}
      \begin{itemize}
      \item<7->{Qualquer objeto que modela o concept \textit{Assignable}}
      \end{itemize}
  \end{itemize}
\end{frame}

\begin{frame}[fragile]
  \frametitle{\textit{Sinais}}
  \begin{itemize}
    \item<1->{on\_change}
      \begin{itemize}
        \item{Alteração em qualquer parte do objeto}
        \item{Disponível em qualquer \textit{observable}}
      \end{itemize}
    \item<2->{on\_insert e on\_erase}
      \begin{itemize}
        \item{Inserção e remoção de elementos de containers}
      \end{itemize}
    \item<3->{on\_value\_change}
      \begin{itemize}
        \item{Alteração no estado de elementos de containers e variants}
      \end{itemize}
  \end{itemize}
\end{frame}

\begin{frame}[fragile]
  \frametitle{\textit{\textit{Ownership} dos \textit{observables}}}
  \begin{itemize}
    \item<1->{Classes}
      \begin{itemize}
        \item{\textit{Ownership} do programador}
      \end{itemize}
    \item<2->{Membros de classes}
      \begin{itemize}
        \item{\textit{Ownership} da classe}
      \end{itemize}
    \item<3->{Elementos de containers e variant}
      \begin{itemize}
        \item{\textit{Ownership} compartilhado}
      \end{itemize}
  \end{itemize}
\end{frame}

\begin{frame}[fragile]
  \frametitle{\textit{Operações básicas}}

  \begin{itemize}
  \item<1->{Obter \textit{Observed} a partir de \textit{Observable}}
    \begin{itemize}
    \item<1->{const Observed\& observable.get()}
    \end{itemize}
  \item<2->{Obter \textit{Observable} de um membro de classe observável}
    \begin{itemize}
    \item<2->{auto\& omember = observable.get<member>()}
      \begin{itemize}
      \item<3->{Retorna uma referência}
      \end{itemize}
    \end{itemize}
  \item<4->{Obter \textit{Observable} do elemento de um container ou variant}
    \begin{itemize}
    \item<4->{auto omember = ocontainer.at(key)}
      \begin{itemize}
      \item<5->{Retorna um std::shared\_ptr para o observável do elemento}
      \end{itemize}
    \end{itemize}
  \item<6->{Obter tipo de um \textit{Observable}}
    \begin{itemize}
    \item<6->{using OFoo = observable::observable\_of\_t<Foo>}
      \begin{itemize}
      \item<7->{Útil para passar \textit{Observables} para funções ou definir visitors de observáveis}
      \end{itemize}
    \end{itemize}
  \end{itemize}
\end{frame}

\end{document}